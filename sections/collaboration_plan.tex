\subsection{Advantage of the collaboration}

% Collaboration Plans for Medium and Large Proposals - Since the success of collaborative research efforts are known to depend on thoughtful coordination mechanisms that regularly bring together the various participants of the project, all Medium proposals that include more than one investigator and all Large proposals must include a Collaboration Plan. While the length of the Project Description for Small and Medium proposals is limited to 15 pages and for Large proposals to 20 pages, for Medium and Large proposals up to 2 additional pages are allowed for Collaboration Plans. Collaboration Plans should be included at the end of the Project Description in a section entitled "Collaboration Plan". The length of and degree of detail provided in the Collaboration Plan should be commensurate with the complexity of the proposed project. Where appropriate, the Collaboration Plan might include: 1) the specific roles of the project participants in all organizations involved; 2) information on how the project will be managed across all the investigators, institutions, and/or disciplines; 3) identification of the specific coordination mechanisms that will enable cross-investigator, cross-institution, and/or cross-discipline scientific integration (e.g., yearly workshops, graduate student exchange, project meetings at conferences, use of the grid for videoconferences, software repositories, etc.), and 4) specific references to the budget line items that support collaboration and coordination mechanisms. If a Large proposal, or a Medium proposal with more than one investigator, does not include a Collaboration Plan of up to 2 pages, that proposal will be returned without review.

% tl;dr - 2 pages on how we'll play nice together

We have proposed here a program of work that brings together two laboratories from different departments and disciplines, with highly complementary expertise, to tackle a novel point of intersection and interaction between the fields of xxxx.
The proposed work requires a team with skills in computer science, engineering, neuroscience, and psychology --- a diverse skillset that the current team is uniquely well-positioned to provide.

Specifically, the skills of the key personnel currently on the team are:

David D. Cox, Ph.D., (co-Principal Investigator) is an Assistant Professor of Molecular and Cellular Biology at Harvard University, and a member of the Harvard Center for Brain Science. He has an extensive background in both computer science and neuroscience. His laboratory's efforts are organized along two concerted fronts: reverse engineering simple biological visual systems (using rodents as a model system), and using resulting knowledge to forward engineer biologically-inspired artificial systems using high performance computing tools.

....

Walter J. Scheirer, Ph.D. (Senior Personnel) is a senior postdoctoral fellow in Prof. Cox's Laboratory.  He is also an Assistant Professor Adjoint at the University of Colorado, Colorado Springs. Previously, he was the director of research \& development at Securics, Inc., an early stage company producing innovative computer vision-based solutions. Dr. Scheirer has extensive experience in the areas of computer vision and human biometrics, with an emphasis on advanced learning techniques. His overarching research interest is the fundamental problem of recognition, including the representations and algorithms supporting solutions to it.


\subsection{Roles of the team members}

The proposed work will be carried out by a highly interdisciplinary team of researchers.  Specific responsibilities within the group are described below.

The Cox Lab team will:

\begin{itemize}
\item do something
\item do something else
\end{itemize}

The xxxx team will:

\begin{itemize}
\item spearhead efforts to do something
\item do something else
\end{itemize}


\subsection{Managerial arrangements}

The co-PIs have extensively discussed and agreed upon the proposed research directions, and [xxxx will do awesome collab stuff together]
The two labs will perform the proposed work in a highly interactive manner, with team members interacting on a day-to-day basis, but with a minimum of bi-weekly meetings attended by both groups.
Both PIs anticipate being coauthors on all papers resulting from the proposed work.
The specific budget provided by each lab for this proposal indicates how project funds will be divided and spent in each group.
Prof. Cox [xxx TODO] will take the lead in administrative interaction with the NSF.
Progress reports will be prepared jointly.

