\documentclass{nsfproposal}
\begin{document}

\section*{Postdoctoral Mentoring Plan}

Funds are requested to support one postdoctoral fellow at Harvard University. The following list outlines a structured mentorship plan to enhance the postdoctoral fellow's academic and professional development:

\begin{itemize}
\item An individually-tailored development plan will be assembled and implemented through meetings between the principal investigator and the fellow. This plan will address both short and long term professional goals, and steps necessary to achieve them.
\item The fellow will meet one-on-one with the PI on at least a weekly basis (but more likely, on a day-to-day basis) to discuss research directions, results, and professional development issues. In addition, the full lab will meet once a week to share results, and participate in our lab journal club.
\item The fellow will be encouraged to attend talks and symposia in the Harvard, MIT, and Boston neuroscience and computer science communities. In addition, the fellow will be given opportunities to meet and network with scientists who are visiting or giving talks at Harvard.
\item The fellow will be encouraged to attend the bi-weekly � ``Soup and Science" lunches at the Rowland Institute (where the principal investigator maintains an informal affiliation), during which students, post-docs and faculty from a diverse range of labs meet to discuss their work.
\item The fellow will participate in grant proposal planning and execution, and will be mentored on the development and maintenance of relationships with industry (e.g. the lab's existing relationships with NVIDIA corporation and Amazon's cloud computing division, both of which provide in-kind computational resources to our group)
\item The fellow will attend at least one major conference to present research results and network with colleagues. The fellow will also naturally participate in the dissemination of research results through the writing of papers and abstracts.
\item The fellow will be encouraged to participate in the array of activities sponsored by the Harvard Office of Postdoctoral Affairs, including orientations, seminars, advising services, and teaching workshops. The fellow will also be encouraged to participate in informal peer initiated networking opportunities, such as the Harvard Postdoc Linked-in group.
\end{itemize}

Success in the implementation of this plan will be monitored periodically through assessment of progress against an individual development plan, assessment of research output (papers, conference presentations, etc.), and interviews with the fellow to assess satisfaction with the mentoring provided.

\end{document}